\chapter{Introdução}
	
\section{Motivação}

\section{Objetivo}

O objetivo deste trabalho é gerar uma solução de software que auxilie no combate à polarização política em redes sociais, em especial a fake news e discurso de ódio. 
Para isso, foi feita uma imersão no contexto político atual do Brasil, através de uma série de conversas e entrevistas com especialistas de ONGs e iniciativas privadas com atuação 
na área, com os quais vislumbra-se acompanhamento durante o processo de desenvolvimento.

Pretende-se desenvolver uma tecnologia que auxilie na moderação de conteúdos, através de modelos de inteligência artificial, integrado ao projeto IBRA USP. O sistema 
desenvolvido terá caráter educativo, com foco em transparência e explicabilidade para o combate a discurso de ódio e fake news. Para isso, serão utilizadas tecnologias ligadas 
a inteligência artificial, modelagem, arquitetura, ingestão e visualização de dados, com alinhamento principalmente para o Twitter. Ademais, haverá emprego de técnicas e 
conceitos de engenharia de software para disponibilizar uma interface simples e intuitiva para os modelos, na forma de um bot para a rede social e uma página web. 

A partir da postagem ou matéria disponibilizada para o robô, fatores relevantes desta, como número de bots engajando, serão indicados para o usuário. 
Considerando essa capacidade de ingestão e classificação de publicações de forma automatizada, o projeto visa ainda à geração de uma base de dados robusta que poderá ser 
utilizada futuramente pela comunidade científica que estuda o tema, conforme descrito em mais detalhes na próxima seção.

\section{Justificativa}
