\documentclass[]{politex}
% ========== Opções ==========
% pnumromarab - Numeração de páginas usando algarismos romanos na parte pré-textual e arábicos na parte textual
% abnttoc - Forçar paginação no sumário conforme ABNT (inclui "p." na frente das páginas)
% normalnum - Numeração contínua de figuras e tabelas 
%	(caso contrário, a numeração é reiniciada a cada capítulo)
% draftprint - Ajusta as margens para impressão de rascunhos
%	(reduz a margem interna)
% twosideprint - Ajusta as margens para impressão frente e verso
% capsec - Forçar letras maiúsculas no título das seções
% espacosimples - Documento usando espaçamento simples
% espacoduplo - Documento usando espaçamento duplo
%	(o padrão é usar espaçamento 1.5)
% times - Tenta usar a fonte Times New Roman para o corpo do texto
% noindentfirst - Não indenta o primeiro parágrafo dos capítulos/seções


% ========== Packages ==========
\usepackage[utf8]{inputenc}
\usepackage{amsmath,amsthm,amsfonts,amssymb}
\usepackage{graphicx,cite,enumerate}


% ========== Language options ==========
\usepackage[brazil]{babel}
%\usepackage[english]{babel}


% ========== ABNT (requer ABNTeX 2) ==========
%	http://www.ctan.org/tex-archive/macros/latex/contrib/abntex2
%\usepackage[num]{abntex2cite}

% Forçar o abntex2 a usar [ ] nas referências ao invés de ( )
%\citebrackets{[}{]}


% ========== Lorem ipsum ==========
\usepackage{blindtext}



% ========== Opções do documento ==========
% Título
\titulo{Democracia Digital}

% Autor
\autor{Fábio Tamaru Nakamura\\
       Henrique Geribello Giabardo\\
       Hugo Martins da Cruz\\
       Jean Lee Bernardes\\
       }

% Para múltiplos autores (TCC)
%\autor{Nome Sobrenome\\%
%		Nome Sobrenome\\%
%		Nome Sobrenome}

% Orientador / Coorientador
\orientador{Prof. Doutor Fábio Levy Siqueira}
\coorientador{Prof. Livre-Docente Roseli de Deus Lopes}

% Tipo de documento
\tcc{de Computação}

% Departamento e área de concentração
\departamento{Engenharia de Computação e Sistemas Digitais}
\areaConcentracao{Ciência de Dados}

% Local
\local{São Paulo}

% Ano
\data{2022}




\begin{document}
% ========== Capa e folhas de rosto ==========
\capa
\falsafolhaderosto
\folhaderosto


% ========== Folha de assinaturas (opcional) ==========
%\begin{folhadeaprovacao}
%	\assinatura{Prof.\ X}
%	\assinatura{Prof.\ Y}
%	\assinatura{Prof.\ Z}
%\end{folhadeaprovacao}


% ========== Ficha catalográfica ==========
% Fazer solicitação no site:
%	http://www.poli.usp.br/en/bibliotecas/servicos/catalogacao-na-publicacao.html


% ========== Dedicatória (opcional) ==========
\dedicatoria{Dedicatória}


% ========== Agradecimentos ==========
\begin{agradecimentos}

Thanks...

\end{agradecimentos}


% ========== Epígrafe (opcional) ==========
\epigrafe{%
	\emph{``Epígrafe''}
	\begin{flushright}
		-{}- Autor
	\end{flushright}
}


% ========== Resumo ==========
\begin{resumo}
Resumo...
%
\\[3\baselineskip]
%
\textbf{Palavras-Chave} -- Palavra, Palavra, Palavra, Palavra, Palavra.
\end{resumo}


% ========== Abstract ==========
\begin{abstract}
Abstract...
%
\\[3\baselineskip]
%
\textbf{Keywords} -- Word, Word, Word, Word, Word.
\end{abstract}


% ========== Listas (opcional) ==========
\listadefiguras
\listadetabelas

% ========== Listas definidas pelo usuário (opcional) ==========
\begin{pretextualsection}{Lista de símbolos}

Lista de símbolos...

\end{pretextualsection}

% ========== Sumário ==========
\sumario



% ========== Elementos textuais ==========

\chapter{Introdução}
	
\section{Motivação}

\section{Objetivo}

O objetivo deste trabalho é gerar uma solução de software que auxilie no combate à polarização política em redes sociais, em especial a fake news e discurso de ódio. 
Para isso, foi feita uma imersão no contexto político atual do Brasil, através de uma série de conversas e entrevistas com especialistas de ONGs e iniciativas privadas com atuação 
na área, com os quais vislumbra-se acompanhamento durante o processo de desenvolvimento.

Pretende-se desenvolver uma tecnologia que auxilie na moderação de conteúdos, através de modelos de inteligência artificial, integrado ao projeto IBRA USP. O sistema 
desenvolvido terá caráter educativo, com foco em transparência e explicabilidade para o combate a discurso de ódio e fake news. Para isso, serão utilizadas tecnologias ligadas 
a inteligência artificial, modelagem, arquitetura, ingestão e visualização de dados, com alinhamento principalmente para o Twitter. Ademais, haverá emprego de técnicas e 
conceitos de engenharia de software para disponibilizar uma interface simples e intuitiva para os modelos, na forma de um bot para a rede social e uma página web. 

A partir da postagem ou matéria disponibilizada para o robô, fatores relevantes desta, como número de bots engajando, serão indicados para o usuário. 
Considerando essa capacidade de ingestão e classificação de publicações de forma automatizada, o projeto visa ainda à geração de uma base de dados robusta que poderá ser 
utilizada futuramente pela comunidade científica que estuda o tema, conforme descrito em mais detalhes na próxima seção.

\section{Justificativa}




\blinddocument


% ========== Referências ==========
% --- IEEE ---
%	http://www.ctan.org/tex-archive/macros/latex/contrib/IEEEtran
%\bibliographystyle{IEEEbib}

% --- ABNT (requer ABNTeX 2) ---
%	http://www.ctan.org/tex-archive/macros/latex/contrib/abntex2
%\bibliographystyle{abntex2-num}

%\bibliography{}


% ========== Apêndices (opcional) ==========
\apendice
\chapter{}
\chapter{Beta}


% ========== Anexos (opcional) ==========
\anexo
\chapter{Alpha}
\chapter{}



\end{document}
